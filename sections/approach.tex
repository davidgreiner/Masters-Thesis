%% LaTeX2e class for student theses
%% sections/approach.tex
%% 
%% Karlsruhe Institute of Technology
%% Institute for Program Structures and Data Organization
%% Chair for Software Design and Quality (SDQ)
%%
%% Dr.-Ing. Erik Burger
%% burger@kit.edu
%%
%% Version 1.3.2, 2017-08-01

\chapter{Approach}
\label{ch:Approach}
The approach to developing an automatic system to detect procedures administered by EMS personnel consists of breaking the procedures into atomical movements, designing an algorithm, and collecting data through a user study.

\section{Hierarchical Task Analysis}
\label{sec:Approach:Hierarchical-Task-Analysis}
EMS personnel can perform many procedures inside an ambulance. The team working on the research grant will automatically every procedure, while this thesis focuses on automatically detecting five procedures: placing a tourniquet, wrapping a wound, bagging, placing an oral airway, and \gls{CPR}. The procedures are broken into their atomical movements using \emph{Hierarchical Task Analysis} \cite{kirwan1992guide}. The resulting atomical movements are further analyzed to include their detectability using commercially available sensors, such as Apple Watch, Myo, Empatic E4, Garmin Watch Forerunner, and Bioharness BT. See Table \ref{tab:hta:cpr} -- \ref{tab:hta:bagging} in the Appendix \ref{chap:appendix} for the full Hierarchical Task Analysis of all procedures. The analysis is done through personal judgment and by talking to EMS personnel at a local fire station.

\section{Algorithm}
\label{sec:Approach:Algorithm}

\subsection{Data Processing}
\label{sec:Approach:Data-Processing}

\subsection{Feature Extraction}
\label{sec:Approach:Feature-Extraction}

\subsection{Machine Learning}
\label{sec:Approach:Machine-Learning}

\section{Data Collection}
\label{sec:Data-Collection}

\subsection{Experimental Design}
\label{sec:Data-Collection:Experimental-Design}

\subsection{Participant Demographics}
\label{sec:Data-Collection:Participant-Demographics}