%% LaTeX2e class for student theses
%% sections/conclusion.tex
%% 
%% Karlsruhe Institute of Technology
%% Institute for Program Structures and Data Organization
%% Chair for Software Design and Quality (SDQ)
%%
%% Dr.-Ing. Erik Burger
%% burger@kit.edu
%%
%% Version 1.3.2, 2017-08-01

\chapter{Conclusion}
\label{ch:Conclusion}

This thesis provided a novel approach to recognize CPR, bag-valve-mask ventilation, placing an oral airway, placing an IV tourniquet, and wrapping a head wound using the Myo. The five procedures were broken into their anatomical movements using hierarchical task analysis, as discussed in Chapter \ref{sec:Approach:Hierarchical-Task-Analysis}. The Myo sensor was picked over other sensors for its ability to sense most of the anatomical movements. The SVM, $k$-NN, and decision-tree machine learning algorithms were compared to Hidden Markov Models for every procedure. The data from the Myo was processed and seven features were generated, as shown in Chapter \ref{sec:Approach:Feature-Extraction}. The machine learning algorithms were evaluated through a user study with ten participants. The study design was presented in Chapter \ref{sec:Experimental-Design:Data-Collection}. The participants were trained in the procedures for two days, and on the third day data was collected for one minute per procedure and five times all procedures sequentially. The machine learning algorithms were compared for their accuracy using the F1 score. The F1 score of the Hidden Markov Model in detecting the procedures was 0.44, only lower to the decision-tree with 0.71. The window size was varied between two seconds and six seconds and resulted in the highest score at six seconds for all machine learning algorithms. The machine learning algorithms were applied to data from every individual participant and compared to data from all participants combined. The F1 score for the procedures were similar within a small margin and the data can therefore be considered generalizable.
\par The F1 score of the HMM was expected to be significantly higher for its ability to detect data sequences. The results from training on individual participants and all participants show that the HMM places second behind the decision-tree.
\par The F1 score of the SVM was lower than expected. Missing computing power for the large dataset resulted in limited ability to vary the parameters to improve the accuracy. 
\par The machine learning algorithms can be further improved by varying the parameters to a greater extend. The low performing SVM algorithm can be improved by varying the $gamma$ and $C$ value, and using a different kernel function.
\par The Hidden Markov Model algorithm can be further improved by optimizing picking the winner of all the models. Currently, the model with the highest the log probability is picked the correct result. The function can be improved by instead running the data again if the log probability of two models are withing a certain margin of error or printing out the probabilities to let the EMS personnel confirm the performed procedure after they return to their base.
\par Additionally, the Hidden Markov Model can be trained supervised. The hierarchical task analysis provides detail about the states a procedure includes. The states can be pre-defined and transition probabilities inferred. Furthermore, deep-learning algorithms, such as neural networks may achieve higher scores.
\par The generalizability can be further improved by collecting data from more participants. The data used in this thesis required participants to use specific hands for every procedure. Data, which includes left- and right-handed participants increases the generalizability.
\par The procedures are performed by EMS personnel on the way to the hospital inside of an ambulance. The movement of the ambulance introduces noise that can effect the accuracy of the machine learning algorithms. Future work can collect data in an ambulance simulation to improve real-life application.
\par Ambulances may include multiple EMS personnel, who can work on the patient simultaneously. The sensor has to be placed on all EMS personnel working on the patient in order to correctly identify the procedure. Future work can collect data from multiple participants at the same time and improve the machine learning algorithm to include data from more than two sensors.
\par Lastly, data from professionally trained EMS personnel may result in more consistent data, which may improve the score of all machine learning algorithms.
\par Overall, in order to use the technology in real world application, such as transmitting treatment information to a hospital in real-time the accuracy needs to be a lot higher. Unreliable treatment information can result in wrong preparation on the hospital side, which may cause permanent injury or result in death. The procedure detection system needs to be further evaluated in real-life non-life threatening scenarios to obtain its effectiveness in improving communication between EMS personnel and hospital trauma staff.