%% LaTeX2e class for student theses
%% sections/conclusion.tex
%% 
%% Karlsruhe Institute of Technology
%% Institute for Program Structures and Data Organization
%% Chair for Software Design and Quality (SDQ)
%%
%% Dr.-Ing. Erik Burger
%% burger@kit.edu
%%
%% Version 1.3.2, 2017-08-01

\chapter{Conclusion}
\label{ch:Conclusion}

This thesis provided a novel approach to recognize CPR, bag-valve-mask ventilation, placing an oral airway, placing an IV tourniquet, and wrapping a head wound using data collected by the Myo and machine-learning algorithms. The five procedures were broken into their anatomical movements using hierarchical task analysis. The Myo sensor was chosen for its ability to sense most of the anatomical movements. Data from the Myo was processed and seven features were generated. The SVM, $k$-NN, and decision-tree machine learning algorithms were compared to Hidden Markov Models for every procedure. The machine learning algorithms were evaluated through a user study with ten participants. The participants were trained in the procedures for two days, and data was collected on the third day to train and validate the machine-learning algorithms. The Decision-tree achieved the highest F1 score and is feasible for inclusion into a system that automatically detects EMS procedures. The machine learning algorithms were applied to data from every individual participant and compared to data from all participants combined. The F1 score for the procedures were similar within a small margin and the data can therefore be considered generalizable.
\par Future work will be able to further segment the procedures to limit the detection to a specific task rather than the entire procedure. Through the series of tasks the procedure can then be implied. The use of artificial neural networks may also improve the detection accuracy and yield better results at detecting the subset of procedures. When higher detection results are achieved more procedures can be included in the detection system.
\par The EMS procedure detection system can already use the developed algorithms to assist EMS personnel in filling out paperwork by suggesting the detected procedure. This allows for faster paper work completion without any harm for the treatment of the patient.