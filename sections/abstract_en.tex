%% LaTeX2e class for student theses
%% sections/abstract_en.tex
%% 
%% Karlsruhe Institute of Technology
%% Institute for Program Structures and Data Organization
%% Chair for Software Design and Quality (SDQ)
%%
%% Dr.-Ing. Erik Burger
%% burger@kit.edu
%%
%% Version 1.3.2, 2017-08-01

\Abstract
Accurate information is vital when transferring patients care between emergency medical services and the hospital trauma team. This thesis proposes an automatic procedure detection system using accelerometer, gyroscope, and electromyography data to recognize a subset of common procedures performed on trauma patients. The following five procedures were chosen for this thesis: cardiopulmonary resuscitation, bag-valve-mask ventilation, placing an oral airway, placing an intravenous tourniquet, and wrapping a head wound. The procedures were then decomposed into their anatomical movements using hierarchical task analysis. The resulting anatomical movements were analyzed to include their sensability using five commercially available sensors: Apple Watch, Myo, Empatic E4, Garmin Watch Forerunner, and Biopac Bioharness BT. The Myo wearable device was chosen for its capability to sense the majority of the anatomical movements. Three machine learning algorithms: decision-tree, SVM, and $k$-NN were trained and compared to a Hidden Markov Model. The features included: mean, standard-deviation, signal magnitude area, root mean squared, and power spectral density. The decision-tree received the highest F1-score of 0.73, followed by HMM with 0.44, $k$-NN ($k=1$) with 0.44, and SVM with 0.33.