%% LaTeX2e class for student theses
%% sections/content.tex
%% 
%% Karlsruhe Institute of Technology
%% Institute for Program Structures and Data Organization
%% Chair for Software Design and Quality (SDQ)
%%
%% Dr.-Ing. Erik Burger
%% burger@kit.edu
%%
%% Version 1.3.2, 2017-08-01

\chapter{Problem Statement}
\label{ch:problem-statement}

Treating a person with severe trauma is extremely challenging, as the injury may result in life threatening effects on blood circulation and tissue oxygenation. Every split-second decision can make the difference between life and death. After the first assessment by the emergency medical services (EMS), the state of the patient has to be monitored constantly. Upon arrival at the hospital, it is important for the trauma team to understand the patient's treatment history, including what medications have been administered and procedures have been performed.  This information is vital to providing the most accurate and beneficial care. Such life-critical information may not be properly communicated when transferring the patient from the EMS personnel to the hospital trauma team. Often EMS personnel rely on their memory to communicate the patient's treatment history to the trauma team, which can be inaccurate. Failing to communicate a complete and accurate treatment history can lead to permanent damage or death. This thesis proposes an automatic reporting system that uses accelerometer, gyroscope, and electromyography (EMG) data to detect five common procedures performed on patients. An algorithm that incorporates machine-learning will be developed to classify the common EMS procedures based on wearable sensor data.