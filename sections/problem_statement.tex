%% LaTeX2e class for student theses
%% sections/problem_statement.tex
%% 
%% Karlsruhe Institute of Technology
%% Institute for Program Structures and Data Organization
%% Chair for Software Design and Quality (SDQ)
%%
%% Dr.-Ing. Erik Burger
%% burger@kit.edu
%%
%% Version 1.3.2, 2017-08-01

\chapter{Problem Statement}
\label{ch:Problem-Statement}

Treating a person with severe trauma is extremely challenging, as the injury may result in life-threatening effects on blood circulation and tissue oxygenation. Every decision can make the difference between life and death. After the first assessment by the emergency medical services (\gls{EMS}), the patient's medical state has to be continuously monitored. Upon arrival at the hospital, it is important for the trauma team to understand the patient's treatment history, including what medications have been administered and emergency procedures have been performed. This information is vital to providing the most accurate and beneficial care; however, such life-critical information may not be properly communicated when transferring the patient from the \gls{EMS} personnel's care to the hospital trauma team. \gls{EMS} personnel often rely on their memory to communicate the patient's treatment history, which can be inaccurate. Failing to communicate a complete and accurate treatment history can lead to permanent damage or death. This thesis proposes an automatic reporting system using accelerometer, gyroscope, and electromyography (\gls{EMG}) data to detect a subset of common procedures performed on trauma patients. An algorithm that incorporates machine-learning will be developed to classify the subset of \gls{EMS} procedures based on the wearable sensor data.
\par Automatically detecting trauma procedures may improve the communication during the care transfer between EMS personnel and the trauma team. Some ambulances are equipped with devices that record the patient's vitals and statistics about cardiopulmonary resuscitation (\gls{CPR}), such as the duration, frequency, and depth of the chest compressions. Missing from these devices is the ability to automatically detect other common EMS procedures. Information about the patient's care is typically entered into the patient's transcript after completing the patient hand-off to the trauma team. When arriving at the hospital, paramedics can provide records e.g., graphs and vital statistics, on a tablet, in addition to the standard oral communication protocol. The full transcript is transmitted via the Internet to the hospital's database when connected to the secured city network. The data transfer usually does not occur until the ambulance arrives back at the station. Currently, transferring patient care information is a slow and static process. Real-time information about the patient's state will be beneficial for the trauma team before the patient arrives at the hospital, as this advanced information knowledge allows the hospital staff to properly prepare the emergency room for the severity of the case.
\par The proposed detection of a subset of commonly performed EMS procedures on trauma patients includes \gls{CPR}, airway management, placing splints, and placing an intravenous (\gls{IV}) line. CPR is the process of helping a person breath using chest compressions and artificial ventilation. Airway management consists of multiple procedures depending on the severity of the trauma. Bag-valve-mask ventilation provides air for patients with breathing difficulties, while an oropharyngeal device is used to manage an unresponsive patient's airway. Oropharyngeal devices keep the tongue from obstructing the airway. Severe trauma patients require intubation, in which a tube is inserted orally and reaches into the windpipe. Splinting stabilizes an extremity that is fractured or broken. An intravenous drip allows for fluids and medication to be administered into the patient’s bloodstream.
\par The EMS procedures are broken into their anatomical movements using hierarchical task analysis, which determined that each EMS procedure requires a specific sequence of anatomical movements. Each sequence will generate patterns in the accelerometer, gyroscope, and EMG data, which the developed task recognition algorithm will be able to detect.
\par A major challenge of detecting a procedure is that the EMS personnel must use two arms to perform the necessary care for the patient. Therefore, each arm must be monitored and both data-sets have to be considered by the automatic task recognition algorithm. A second challenge is that there are individual differences in the body movements when executing EMS procedures. Depending on the EMS personnel, body movements vary from using a different finger to another motion. Another challenge is accounting for an ambulance's abrupt turns, fast acceleration and sudden stops, as such ambulance movements generate noise on the accelerometer and gyroscope data unrelated to the EMS personnel's movement, and must be filtered.
\par Chapter \ref{ch:Literature-Review} provides background information on existing systems to detect physical movement. Chapter \ref{ch:System-Design} proposes an algorithm that detects trauma procedures by \gls{EMS} personnel. Chapter \ref{ch:Experimental-Design} introduces an experimental design for studies to collect data and test the algorithm. Chapter \ref{ch:Results} presents the results of the studies. Finally, Chapter \ref{ch:Conclusion} outlines the contribution and drawbacks of the algorithm, followed by a discussion how future work could improve the system. 