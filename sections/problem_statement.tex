%% LaTeX2e class for student theses
%% sections/problem_statement.tex
%% 
%% Karlsruhe Institute of Technology
%% Institute for Program Structures and Data Organization
%% Chair for Software Design and Quality (SDQ)
%%
%% Dr.-Ing. Erik Burger
%% burger@kit.edu
%%
%% Version 1.3.2, 2017-08-01

\chapter{Problem Statement}
\label{ch:Problem-Statement}

Treating a person with severe trauma is extremely challenging, as the injury may result in life threatening effects on blood circulation and tissue oxygenation. Every split-second decision can make the difference between life and death. After the first assessment by the emergency medical services (\gls{EMS}), the state of the patient has to be continuously monitored. Upon arrival at the hospital, it is important for the trauma team to understand the patient's treatment history, including what medications have been administered and procedures have been performed.  This information is vital to providing the most accurate and beneficial care. Such life-critical information may not be properly communicated when transferring the patient from the \gls{EMS} personnel to the hospital trauma team. Often \gls{EMS} personnel rely on their memory to communicate the patient's treatment history to the trauma team, which can be inaccurate. Failing to communicate a complete and accurate treatment history can lead to permanent damage or death. This thesis proposes an automatic reporting system that uses accelerometer, gyroscope, and electromyography (\gls{EMG}) data to detect five common procedures performed on patients. An algorithm that incorporates machine-learning will be developed to classify the common \gls{EMS} procedures based on wearable sensor data.\\
Automatic detection of trauma procedures seeks to improve the communication during the care transfer between EMS personnel and the trauma bay team. Currently, oral communication is used when arriving at the hospital, while the transcript is later uploaded to the hospital's database. Some ambulances are equipped with devices that can record vitals and cardiopulmonary resuscitation (\gls{CPR}). Missing in these systems are other procedures and the detection of administered medication, which are entered into the patient's transcript by hand. This thesis proposes the detection of five common procedures on trauma patients: \gls{CPR}, placing an oral airway, bag-valve-mask (\gls{BVM}) ventilation, placing splints, and placing an intravenous (\gls{IV}). Cardiopulmonary resuscitation is the process of helping a person breath using chest compressions and artificial ventilation. The procedure of placing an oral airway is mostly used on unresponsive patients without a gag reflex. An oral airway keeps the tongue from obstructing the airway. Bag-valve-mask ventilation manages airways; it provides air for patients with breathing problems. Splinting is a procedure administered to stabilize a body part that is fractured or broken. An intravenous drip allows the administration of fluids, medication, and blood into the patient’s bloodstream.\\
The basis for the automatic detection is an algorithm that uses accelerometer, gyroscope, and \gls{EMG} data. Detected physical movement of \gls{EMS} personnel performing a procedure will be recorded and transmitted to the trauma bay team.\\
Initially, data will be collected from wearable sensors during a user study. This data is used to implement and train an algorithm incorporating machine-learning. A user evaluation will be conducted to test the algorithm's accuracy in detecting physical movement and procedures.\\
Chapter \ref{ch:Literature-Review} provides background information on existing systems to detect physical movement. Chapter \ref{ch:Systems} proposes an algorithm that detects trauma procedures by \gls{EMS} personnel. Chapter \ref{ch:Experimental-Design} introduces an experimental design for studies to collect data and test the algorithm. Chapter \ref{ch:Results} presents the results of the studies. Finally, Chapter  \ref{ch:Conclusion} outlines the contribution and drawbacks of the algorithm, followed by a discussion how future work could improve the system. 