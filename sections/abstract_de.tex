%% LaTeX2e class for student theses
%% sections/abstract_de.tex
%% 
%% Karlsruhe Institute of Technology
%% Institute for Program Structures and Data Organization
%% Chair for Software Design and Quality (SDQ)
%%
%% Dr.-Ing. Erik Burger
%% burger@kit.edu
%%
%% Version 1.3.2, 2017-08-01

\Abstract
Genaue Informationen sind wichtig, wenn die Verantwortung für Patienten von Rettungssanitätern im Krankenwagen an Krankenhauspersonal übertragen wird. Diese Masterarbeit schlägt ein automatisches Behandlungserkennungs-System vor, welches durch Beschleunigungs-, Orientierungs- und Muskeldaten einen Teil von üblichen Behandlungen an Traumapatienten erkennt. Für diese Arbeit wurden die folgenden fünf Verfahren ausgewählt: kardiopulmonale Reanimation, Masken-Beatmung, Einführung eines oralen Luftweges, Platzierung eines Venenstauer und Bandagieren einer Kopfwunde. Die Verfahren wurden unter Verwendung einer hierarchischen Aufgabenanalyse in ihre anatomischen Bewegungen zerlegt. Fünf handelsübliche Sensoren:  Apple Watch, Myo, Empatic E4, Garmin Watch Forerunner und Biopac Bioharness BT wurden dahingehend analysiert, wie genau die anatomischen Bewegungen erkannt werden können. Das am Unterarm tragbare Myo-Gerät wurde aufgrund seiner Fähigkeit ausgewählt, die meisten anatomischen Bewegungen zu erfassen. Drei maschinelle Lernalgorithmen: Entscheidungsbaum, SVM und $k$-NN wurden trainiert und mit einem Hidden-Markov-Modell verglichen. Die Features umfassten: Mittelwert, Standardabweichung, Signalbetragsbereich, quadratischer Mittelwert und spektrale Leistungsdichte. Der Entscheidungsbaum erreichte den höchsten F1-Wert von 0.73, gefolgt von HMM mit 0.44, $k$-NN ($k=1$) mit 0.44 und SVM mit 0.33.